\documentclass{article}

\usepackage{fontspec}
\usepackage[latin, british]{babel}
\usepackage[backend=biber, style=authoryear-comp, autocite=inline, sorting=nyt, sortcites=true, maxcitenames=2, natbib=true, maxnames=100]{biblatex}
\usepackage[style=british, thresholdtype=words, threshold=38]{csquotes}
\usepackage[hidelinks]{hyperref}
\usepackage{cleveref}
\usepackage[backgroundcolor=pink,bordercolor=pink,textsize=scriptsize]{todonotes}

% References
\addbibresource{neagl.bib}

\SetCiteCommand{\autocite}

% Text expansion
\newcommand{\dash}{\unskip{—}}

% Foreign languages
\newcommand\foreign[2]{\foreignlanguage{#1}{\emph{#2}}}

\newcommand\texttitle[1]{\emph{#1}}

\setmainfont{Hoefler Text}

\title{Non-essentialist, activity-grounded lifeforms}
\author{Barnaby R. Hutchins}

\begin{document}

\maketitle

\begin{abstract}
If we're looking for the grounds for a notion of lifeform in Spinoza's system, it might seem that the natural place to look, given his ontology, would be in the morphology of bodies, with action and ethics in turn grounded in that lifeform. Spinoza rejects anything like substantial forms that might be capable of imparting something like a species essence to individual things; if there's a difference between, e.g., humans and horses to be found, it makes sense to look for it in the seemingly obvious morphological differences. On such an interpretation, a horse has, and acts on, a specifically \textquote[E3p57s]{equine lust} just because it has a specifically equine morphology; similarly, humans are what's most useful for humans, and the activities of a human maximally \enquote{agree}  with the activities of other humans (E4p35c1), just because of their shared human morphology. A morphology-grounded reading, however, runs into serious issues when faced with Spinoza's treatment of bodies, chiefly because his ontology lacks the resources for a sufficiently robust speciation of morphologies.

In this chapter, I argue instead for an activity-grounded conception of lifeform. On this conception, lifeforms are constituted through agreement in activity – it's not that human activities agree \emph{because} they are grounded in a human morphology; rather, it's that certain activities are human \emph{just insofar as} they all agree. The activities at stake are multiply realizable through the morphological level, allowing for a notion of lifeform that's compatible with Spinoza's gradualism about morphology. Agreement in activity provides Spinoza with a concrete (if, admittedly, underexplained) ground for lifeform, and avoids the issues that come from looking to morphology to do the job. I argue that an activity-grounded reading tracks Spinoza's use of lifeform-related concepts, and that it results in a treatment of lifeforms as dynamic, mutable, and non-essentialist.
\end{abstract}

\section{Introduction}

In action-theoretic \dash and, broadly speaking, biological \dash contexts, lifeforms are invoked in order to explain, in various ways, what a thing does.
Humans do what they do, in the ways that they do, because of an underlying lifeform. Ants do very different things, very differently, from us because their lifeform differs significantly from ours. The lifeform is the background against which actions make sense: it makes a lot of sense for a human to, say, change a light bulb; for an ant, that would be a little strange. This can be cashed out in various ways \dash in terms of the role of artificial light in current human life, in terms of our opposable thumbs that allow us to actually get hold of the thing to change it, and so on. In these contexts, when we talk about lifeforms, we talk about them as the grounds of actions.

There are some good reasons to see something like this lifeform concept as being at work in Spinoza's system (even though he doesn't use the term itself\footnote{I use \enquote{lifeform} here \dash a term that Spinoza never uses \dash because this chapter is ultimately concerned with the action-theoretic context. To the extent to which my reading works for lifeform, though, it should extend more or less straightforwardly to certain senses of \enquote{species}, and somewhat less straightforwardly, \enquote{form} \dash terms that Spinoza does \emph{occasionally} use in appropriate contexts.}), especially when he makes claims such as, \textquote[E4App\RN{9}]{\textins*{n}othing can agree more with the nature of any thing than other individuals of the same species}, or many of his references to \enquote{form}. But Spinoza denies real universals, denies that commonalities pertain to the essences of singular things (E2P37), and endorses gradualism about bodies (i.e., there are no hard boundaries between natural kinds, and seemingly no ontological resources to institute them).

So there is some puzzle about the role of something like a lifeform in what we might broadly, and maybe inaccurately, call Spinoza's philosophy of action \dash a puzzle that warrants addressing. That's an issue of what the notion of lifeform can do for a reading of Spinoza; there is also the issue of what Spinoza can do for the notion of lifeform.

Work on lifeforms in philosophy usually starts from a broadly Aristotelian (or, in some cases, Hegelian) framework, and usually assumes some non-trivial degree of essentialism: humans do humane things becomes of some underlying essential human lifeform; sharks do sharkish things because of some underlying essential shark lifeform; etc. Analysis of action in terms of lifeform is useful, but it would be nice if it could come without essentialist commitments attached.

Spinoza's commitments to gradualism and merely nominal universals more or less rule out essentialism for something-like-a-lifeform. This is precisely the reason that the role played by lifeforms in Spinoza's system is so difficult to pin down: intuitively, lifeforms seem as though they should entail essence; to seek them out in a non-essentialist context seems like a category mistake. So if Spinoza does indeed have a coherent conception of lifeform that is capable of playing an appropriate action-theoretic role, it breaks the expected entailment from lifeform to essence. And, if so, Spinoza gives us a non-essentialist way of addressing and employing lifeforms. For exactly the reason that the issue of lifeform in Spinoza's system is tricky, it is a valuable resource.

% What are the features of a non-essentialist notion of lifeform?\todo{Fill out}
\section{Grounding lifeforms}\label{sec:GroundingLifeforms}

In the Aristotelian framework, you get lifeforms for free, as part of the basic ontology. Consequently, you don't need to find something else to which to reduce them. In a more austere framework like Spionoza's, by contrast, you won't find anything like lifeforms anywhere within the fundamental ontology: you have to build them out of more fundamental entities; whatever a lifeform is, it's going to have to be reducible to something else.

The obvious place to look for a reductionist account is morphology \dash something about the structure of bodies gives them the kind of lifeform they have. This accords with intuitions about, e.g., opposable thumbs and light bulbs. It also fits with Spinoza's explanation of \textquote[E2p13s]{the difference between the human mind and the others, and how it surpasses them} in terms of the complexity of the human body: there's seemingly something about the constitution of the human body that differentiates humans from other lifeforms.

The most likely candidates for a morphological reading seem to be either morphological complexity or extrinsic denomination (i.e., as the \textquote[E4Pref; G \RN{2}/208]{model of human nature which we may look to}). The former seems insufficiently robust (especially for the sake of the explanatory role that lifeform takes in action theory and ethics): what determines the degree of complexity at which one lifeform turns into another? This is the gradualism problem in a nutshell.\footnote{Arguably, Spinoza could get one lifeform \dash the human lifeform \dash out of complexity, as its upper limit (or two, if we also count the lower limit as a lifeform), without falling prey to the gradualism problem \dash assuming that there is a way for the bounds of complexity to be determinable. Presumably, they would have to be determined relatively, as the most (or least) complex amongst all actual things. But, (a) complexity, even when couched in terms of being capable of \textquote[E2p13s]{doing many things at once, or being acted on in many ways at once}, is multifarious, and sufficient reason would have to be found for preferring one form of complexity over others. Moreover (b), any given human is going to differ to some extent in complexity from any other, such that, even if an upper limit to complexity can be determined, it will only pick out one individual. If we want a shared lifeform, we're back to appealing to an underdetermined region in the complexity continuum.} It also contravenes Spinoza's proscription of commonalities in the essences of singular things.

The extrinsic option has the advantage of embracing the ad-hocness that would undermine a complexity account: I can stipulate that things that look roughly roughly like \emph{this} \dash say, have two legs and no feathers \dash are humans; you might disagree. In which case, the grouping under the lifeform \enquote{human} itself is grounded in my mind rather than in humans themselves. This might well be perfectly adequate for analytical means \dash pragmatic groupings can be useful analytic tools \dash but insufficient for lifeforms, understood in this way, to have an explanatory role: we cannot say $x$ $\phi$s \emph{because} it is human, if our criteria for grouping $x$ as human bottom out at \enquote{because I said so}.

I am not going to argue that there is no workable way whatsoever to ground a notion of lifeform in morphology for Spinoza; indeed Tropper's chapter\todo{I think! I haven't read the latest version yet. Sorry, Sarah, if I'm making stuff up here} in this volume goes somewhat in that direction. Rather, I am going to argue for an alternative possibility: there are useful resources for grounding lifeform to be found in activities \dash the stuff that singular things \emph{do} \dash and an activities-grounded, or activity-priority, reading fits well with the textual evidence and the ways in which Spinoza employs something-like-a-lifeform. Moreover, and ultimately more germane to my purposes here, an activity-priority reading is exactly the sort of reading that allows for a realist, non-essentialist conception of lifeform, in which lifeforms get to be real and non-trivial aspects of the world, while still being fundamentally mutable.

Trying to talk about what something \emph{does} is oddly tricky in Spinoza's system. Spinoza talks about \enquote{actions}, but, strictly speaking, Spinozistic actions can only be \enquote{doings} that are grounded in reason; \enquote{action} is too narrow a category for my purposes here. Conceptually, \enquote{motions} would probably do the job, but the term is intuitively jarring: intuitively, we tend to think of motion purely in terms of local motion; this chapter is about more than just local motion, so what's connoted is again too narrow, even if \enquote{motion} does denote the correct set of things in principle. It's for these reasons that I've gone with \enquote{activities}. By \enquote{activity}, I mean simply what things do, or modal change, and I mean to avoid the restrictive implications of \enquote{actions} and \enquote{motions}, while incurring as little linguistic awkwardness as possible.

It might seem strange to ground something like lifeform in something like activity. Typically, we think of activities as derivatives of more fundamental entities. We tend to think of them as \emph{changes} in \emph{bodies}, and we tend to think that the bodies persist while their activities change. On this understanding, activities are both ontologically posterior to bodies and, more or less, ephemeral, making them a seemingly rather poor choice as grounds for anything much at all.

More specifically, and perhaps tellingly, we tend to think of activities as \dash or on the model of \dash modes of substances. If we think of bodies as substances, then of course we would expect bodies to be grounds. And, if a lifeform is some kind of constraint to activity, of course the notion of activities as grounds would look incoherent. Within a lot of frameworks, the substantival model of body–activity relations is apt enough: it obviously works for the Aristotelian framework, is canonical for the Cartesian framework, and seems to track most materialisms and physicalisms.

It's not clear, though, that this is an intuition that ought to apply to Spinoza's system, where bodies and activities are equally modal. While an argument could presumably be made that (modal) activities are still supposed to be only derivative of (modal) bodies, such that there are necessary and consistent intramodal priority relations between bodies and activities, there does not seem to be anything in Spinoza's system to compel that reading; nor does there seem to be much to support such a reading, beyond trying to preserve our intuitions. Indeed, if conatus is understood dynamically, then such priority relations cannot obtain (see \cref{sec:Agreement}). Consequently, for the sake of this chapter, I'll assume that bodies and activities are modes fundamentally on the same level, and that, whatever a lifeform might be, it can just as plausibly be grounded in activity as in body.

% I am also not going to go into the much broader issue of essences in general in Spinoza's system. The question at stake in this chapter is, if there is something like a lifeform, in particular, in the system, what does it consist in? What can we point to in the world and say \enquote{that right there: that's a lifeform}? There are two gotchas in trying to answer that question. One is Spinoza's denial of real universals: there is no entity that is distributed throughout all humans that makes them human. So, if we are to find a lifeform, we can find it only in the particulars. Second, there is Spinoza's gradualism about bodies: all change happens along a continuum, with no hard boundaries between one kind of body and another; that is, there are no real natural kinds of body. This is a problem because the obvious answer to the question of lifeform would be something about the morphology of bodies, but if the morphology of bodies is continuous, it is not clear what ontological resources there could be to build up a sufficiently robust notion of lifeform on that basis.

There are at least two, related, issues raised by this use of \enquote{activity}. First, if an activity is what something does, this sounds suspiciously substantival: there is some \emph{thing} that does some \emph{activity}. If what we're concerned with is specifically the activity \emph{of} things, then the things look very much as though they ought to be the grounds again. Here, I argue that, while particular activities are necessarily in relation with particular bodies, and while the particularity of both is salient to the role of lifeform, this does not in any way mean that the bodies involved have ontological priority over the activities involved.

The second issue is that, if the term \enquote{activity} allows for a wider scope than \enquote{action}, then it looks as though we are allowing externally-determined motions into the picture, which seems to go against at least some of what Spinoza says about activity in relation to something-like-a-lifeform. This is a feature, not a bug. (And I argue as much in \cref{sec:Mutuality}.) To assume that the relevant activities for grounding a lifeform are only those that are internally determined is to beg the question on the essentialist's behalf: it's to assume that only what follows from the essence of a thing is informative about its lifeform.

\section{Lifeform and agreement}\label{sec:Agreement}
Spinoza uses something like a notion of lifeform most explicitly in Part Four of the \texttitle{Ethics}, in the context of the discussion of inter-human relations. In the appendix to Part Four, he claims that \blockquote[E4app\RN{9}]{\textins*{n}othing can agree \textins{\foreign{latin}{convenire}} more with the nature of any thing than other individuals of the same species}. A natural reading of this passage \dash that is, a natural reading given substantivalist intuitions \dash might be that there is some phenomenon called \enquote{agreement} that is grounded in some property called \enquote{species}: on this reading, \enquote{species} preexists, and is more fundamental than, \enquote{agreement}, such that we can have species without actual agreement (say, if the members of the species are dispersed here and there without meeting).

What I am proposing here is an inversion of the grounding relations in the above reading: it's not that species grounds agreement; rather, agreement is what grounds species. To (metaphysically) explain what species \dash or, in the terms I'm using here, lifeform \dash is, we look to agreement, rather than the other way around. Read in this way, Spinoza's point is not that these particular things agree most with each other because they are all elements of the same species-set, but that they are elements of the same set of things just \emph{because} they maximally agree: membership status is established in terms of agreement, and a lifeform just is a set of things that maximally agree (in some as-yet-unspecified way). This is what an agreement-priority conception of lifeform consists in.

Below, I argue that we can define agreement in terms of usefulness, such that we can identify the members of a species or  lifeform as those singular things for which a given set of activities is maximally useful: Let's say that there are some activities, performed in certain ways, that are more useful to me than others are; those activities constitute my lifeform. And if the same activities are also of maximal utility to you, then we share a lifeform.

There does not seem to be anything in E4app\RN{9} that compels taking one interpretation over another, substantival intuitions aside. But the advantage of the agreement-priority reading over the species-priority reading is that it provides a means of addressing the thorny species problem, rather than simply throwing us back into the thorns. It does, however, raise the issue of what exactly agreement might be: if a species is a set of things that maximally agree, what is it that they maximally agree in, or, for that matter, on?

\section{Agreement and usefulness}\label{sec:Use}

What E4app\RN{9} is summarising is E4p35c1: \blockquote[E4p35c1]{what is most useful to man is what most agrees \textins{\foreign{latin}{convenit}} with his nature (by P31C), that is (as is known through itself), man}. The overall point here is broadly the same as in the passage from E4app\RN{9}, albeit restricted to the human case. But it adds an extra layer: usefulness. The relationship between usefulness and agreement could again be read in a few different ways: (a) the agreement could ground, or explain, the usefulness; or (b) the usefulness could ground the agreement; or (c) the two could be mutually implicated, or ultimately ontologically really identical. I'm going to argue for the latter, and for the added bonus that Spinoza's discussion of usefulness is informative about agreement.

Now, admittedly, (a) does, on the face of it, look like the more promising option in light of the text: the discussion of agreement in E4p35 is tied to reason, and reason is ultimately tied to a thing's own essence, such that, when a thing acts from reason, it acts from what's inherent to it (E3p4–7). On top of that, in the second corollary to E4p35, Spinoza explicitly equates usefulness to egoism: \textquote{\textins*{w}hen each man most seeks his own advantage for himself, then men are most useful to one another}, where a person's seeking their own advantage consists in \textquote{acting according to the laws of his own nature}. This does sound as though usefulness is being explained in terms of agreement, and agreement in terms of similarity in some pre-existing essence.

However, the essence, or nature, in question here is the \enquote{actual essence} of the thing, or its conatus (E3p7).\footnote{I'm concerned here only with actual essence, and am leaving aside the issue of whether there is a \enquote{formal essence} beyond the actual essence. See, e.g., \cites{Laerke:2017,Nadler:2012,Ward:2011}.}
The conatus of a thing is active and dynamic; it's the \textquote[E3p7]{power of each thing, \emph{or} the striving by which it (either alone or with others) does anything, or strives to do anything}, which is merely \enquote{distinguished by reason, or rather verbally}, and is not \textquote[CM 1/6; G \RN{1}/248; {\cite[314]{C1}}]{in any way really distinct}, from the thing, or the essence thereof, itself.\footnote{On the dynamic reading of the conatus, see, e.g., \cites{Williams:2019}[][Part 4, ch. 13]{RenzExplainability}; for contrasting readings, see, e.g., \cites{Bennett:1984}{DellaRocca:1996}{Viljanen:2011}.}\todo{Add more refs}

The point here is that a thing's actual essence is not ontologically distinct from its activity: there is no pre-existing essence that subsequently determines activity; rather, acting in such-and-such a way just is having such-and-such an essence, and having such-and-such an essence just is acting in such-and-such a way. In which case, the actual essence cannot be ontologically, or logically, prior to the activity, and neither of the two can be \emph{the} explanation of the other \dash what you are does explain what you do, but only insofar as what you do explains what you are.

This mutual implication structure is replicated in E4p31c, which is the passage referenced in E4p35c1 in relation to agreement and usefulness: \blockquote[][.]{\textins*{f}rom this it follows that the more a thing agrees with our nature, the more useful, \emph{or} better, it is for us, and conversely, the more a thing is useful to us, the more it agrees with our nature} This is a reasonably clear statement of mutual implication: it tells us that \emph{if greater agreement, then greater utility} is equivalent to \emph{if greater utility, then greater agreement}. In itself, though, the passage is fairly inconclusive with respect to ontological identity \dash assuming $p \iff q$, the statement will hold without needing to take $p$ and $q$ to be really identical. But, in light of conatus–actual-essence identity, there seems to be more than just biconditionality at stake here.

E4p31d explicitly refers back to the conatus doctrine (through E3p6), as the basis from which the corollary follows: i.e., given the account of the conatus, it follows that usefulness and agreement are mutually implied. Moreover, E4p31 correlates agreement not just with usefulness (and vice versa), but also with goodness: E4p31 itself reads \textquote{\emph{\textins*{i}nsofar as a thing agrees with our nature, it is necessarily good}}. This initial correlation between agreement and goodness turns into a statement of identity in E4p31c: \textquote[my emphasis]{\textelp{} that which agrees with our nature, \emph{that is} (by P31), \textelp{} the good}. Given that Spinoza justifies the identification of agreement and goodness by appeal to E4p31 itself, it seems as though the implication between agreement and goodness in the proposition is to be understood as an identity rather than a mere correlation: the good just is what agrees with our nature, and what agrees with our nature just is the good.

This identity statement is then used as the basis for a restatement of the correlation between agreement and usefulness: the corollary continues, \textquote[E4p31c; my emphasis]{\textins*{n}othing, therefore, can be good except insofar as it agrees with our nature. So the more a thing agrees with our nature, the more useful it is, \emph{and conversely}, q.e.d.}. Evidently, the good is identified with agreement, which is, at the very least, correlated with usefulness. This fits in well with Spinoza's definition of the good: \textquote[E4D1; my emphasis][.]{\textins*{b}y good I shall understand what we certainly know to be \emph{useful} to us} As Spinoza characterises it, then, we have good reason to take goodness to be really identical with usefulness.

In the CM, this understanding of goodness is also specifically linked to the conatus, where Spinoza criticises those who \textquote{labour under a false prejudice} in taking there to be such a thing as an absolute, \textquote[CM 1/6; G \RN{1}/248; {\cite[314]{C1}}]{Metaphysical good}. This is the same passage as mentioned above, in which Spinoza classifies the distinction between a thing and its conatus as merely verbal: the real identity of thing and conatus is emphasised there in order to defend the position that there is no good beyond what a thing strives for (CM 1/6; G \RN{1}/247; {\cite[313]{C1}}) \dash to take the good to be ontologically distinct from a thing's conatus, and from the thing itself, is to \textquote[CM 1/6; G \RN{1}/248; {\cite[314]{C1}}][.]{confuse a distinction of reason with a real or modal distinction}

From E3p7, we can say that a thing's actual essence and its conatus are identical. From CM, we get that goodness is merely verbally distinct from both a thing's conatus and the thing itself. E4D1 identifies goodness with usefulness. And E4p31 identifies goodness with agreement. What this gives us is a variety of transitive routes for concluding that usefulness and agreement are indeed really identical \dash and are also identical to goodness, conatus, and actual essence.

Perhaps this doesn't sound very helpful: we were looking to usefulness and agreement as means to ground a notion of lifeform, and we ended up with the actual essence of a singular thing, and with a whole lot of seemingly useful distinctions being eroded into identity. In other words, we seem to be right back at the problem of how to build up a notion of lifeform when all the available ontological resources are singular things.

However, just because some things are really identical, that doesn't mean that the rational distinctions between them are uninformative.\footnote{A distinction of reasoned reason would be genuinely explanatory; a distinction of reasoning reason, merely analytic. Given Spinoza's \textquote[CM 1/6; G \RN{1}/248; {\cite[314]{C1}}]{or rather verbally} comment, the latter seems the more likely option. Both should, I think, work here regardless.}
If we talk purely in terms of a singular thing's actual essence, say, we seem to be talking about something purely intrinsic. The same might, maybe somewhat less easily, be said of conatus.\footnote{A dynamic conception of conatus rules out exactly this ascription of pure intrinsicality: \textcquote[123]{Williams:2019}[.]{the \emph{conatus} principle is an essential characteristic of \emph{all} things and is most usefully conceived \emph{outside} the subject, in the wider context of an ontology of relation} Understood dynamically, conatus has unrestricted arity; understood statically, it is nullary (see \cref{fn:Arity}).}
But all this becomes far harder to treat as purely intrinsic if we are looking through the lens of usefulness, agreement, or even the good, all of which seem to have non-trivial relational dependencies \dash \emph{this other thing} is useful to me, agrees with me, or is good for me.

\section{Maximality and mutuality}\label{sec:Mutuality}

In Spinoza's strong statement of something-like-a-lifeform in E4p35, he focuses on humans insofar as they \textquote[E4p35d; my emphasis]{\emph{do} only those things which are good for human nature}. The concern here is with what humans \emph{do} that is good \dash ultimately, maximally good \dash for human nature. Most of the discussion that follows E4p35 focuses on what humans do when they act from reason \dash that is, actions, specifically, rather than the broader category of activities. Reason is indeed going to have to be significant for the notion of lifeform, at least inasmuch as it constitutes the active, or agential, side of the notion (which is especially relevant for an ethics).\todo{I probably ought to go into more detail about reason for the final draft} But the Spinozistic lifeform also has a patientive side: on Spinoza's account, a lifeform does not consist purely in what an agent does in isolation; what counts is how what's done by agents affects others.

This much is already implied in the emphasis on agreement in E4p35. But it comes out more explicitly in the scholium, where Spinoza switches to a focus on the passive side of things, specifically in the context of human society and the interactions it entails. He points out that \textquote{the society of our fellow men} is useful to us (we get \textquote{many more advantages than disadvantages} from it), and that, \blockquote[][.]{\textins*{m}en still find from experience that by helping one another they can provide themselves much more easily with the things they require, and that only by joining forces can they avoid the dangers which threaten on all sides \dash not to mention that it is much preferable and more worthy of our knowledge to consider the deeds of men, rather than those of the lower animals} The sociality illustrated here, along with the underscoring of the passive element, is already present in the account of the conatus in E3p7, where Spinoza mentions \textquote[E3p7d; my emphasis]{\textelp{} the striving by which it (either alone \emph{or with others}) does anything}. Of those activities done by singular things, some agree with me, insofar as they are useful to me.

It is not that I do certain activities from my own inherent nature, that other humans do certain activities from their own inherent natures, and that there is some shared set of properties, underlying our individual natures, that constitutes a general human nature, such that our actions are done independently but happen to agree because they derive from that shared nature.\footnote{\label{fn:Arity}As \citeauthor{Illari&Williamson:2013}, writing in a different context, put it, activities have \textcquote[72]{Illari&Williamson:2013}{unrestricted arity} \dash they can be applied to however-many operands. Spinozistic lifeforms seemed to have similarly unrestricted arity; things acting in isolation from purely inherent natures, on the other hand, would be nullary.}
Rather, some activities agree with me, and just insofar as some things produce those activities, and just insofar as I produce equivalently useful activities for those things, those things and I share a nature, or participate in the same lifeform.

Of course we \dash and I'll restrict the phrasing to the human case here, for the sake of clarity, but the same should apply for any singular thing with a lifeform \dash are acted upon by a multitude of different things. Many, probably most, of those things are going to be non-human: just because a thing is useful to me, that doesn't mean we share a lifeform. Many things \dash water when I'm dehydrated, a blanket when it's cold \dash seem to be good for me without, I take it, sharing my lifeform. So while everything that acts on me will presumably have to be involved in my own essence, a non-vacuous notion of lifeform will need to be more restrictive.

I take it that instituting such restrictions is precisely what's going on in passages such as the discussion in E4p35: what gets included in the human lifeform are those things for which a particular set of activities is most useful. Thus, all those singular things for which humane activities are most useful are what we call \enquote{human}. By contrast, a blanket is very useful for keeping me warm, which is indeed a very useful thing to do, but its usefulness doesn't extend much beyond that; having only a limited degree of usefulness to me, of agreement with me, the blanket doesn't share my lifeform.

Furthermore, I don't offer much utility to a blanket; if I happen to further its conatus, it's more or less incidental (there's little I do for a blanket \dash perhaps I wash it from time to time, but that's probably more to its own detriment than anything, damaging its fibres and its weave until it eventually falls apart). Humans, on the other hand, are precisely those things which do a lot that's useful for me, and for which I do a lot that's useful\footnote{Which is absolutely not to say that I, specifically, am a great boon to humanity (or rather, any greater than anyone else, since any human is effectively defined as that which is a boon to humanity) \dash just that the things I do qua human are by definition things that are useful to humans. The more natural way to express this would be in terms of what I, or humans, \emph{can} do, but that phrasing would imply that capacities are at stake; for an activity-priority reading to work, it has to rely on actual activities.} \dash not becomes of some prior essence, but because, on this conception, humans are defined as \emph{whatever fits that description}.

There are cases of greater mutual usefulness than in the blanket case. Humans might well interact with, say, trees, and might well do so in broadly mutually beneficial ways (trees might provide humans with fruit or wood, or simply oxygen; humans might plant or graft trees, protect them from disease, provide them with carbon dioxide, and, by eating their fruit, distribute their seeds; and so on). The case is somewhat less asymmetrical as compared to the blanket case, but the utility of trees for humans is still within a fairly restricted domain, so tree-ish activities are unlikely to be maximally useful to humans. By the same token, humane activities are probably not going to be maximally useful to trees (we cut them down to obtain material that's useful to us, after all).

In order to properly restrict what is included in a lifeform, then, and to get the reciprocality implied by \textquote[E4p35c1; my emphasis]{there is nothing more useful to \emph{man} than \emph{man}} right, the utility needs to be not only maximal or mutual, but \emph{mutually maximal}.

To be clear, the point here is not that there is some set of singular things called humans that are predisposed to maximally benefit from some set of humane activities. Rather, the set of humans is picked out by the set of activities in which they mutually find maximal utility. There needn't be any separate, particular, pre-existing property in the singular thing in virtue of which it benefits most from certain activities in order for a notion of lifeform construed in this way to work: satisfaction of maximal usefulness is, presumably, going to be multiply realisable at, say, the morphological level \dash as I go over in more detail in \cref{sec:ObjMorphology}.

In this way, agreement gets cashed out not in terms of shared common properties, but in terms of a distinct state of affairs that obtains \emph{at} each member of the lifeform, but \emph{in relation to} other members of the lifeform. When some singular thing $x$ finds mutual maximal usefulness with $y$, $z$, \dots{}, that is a determinate state of affairs for $x$, rather than some universal that transcends $x$. This produces a notion of lifeform that quantifies across all its members without contravening the interdiction against commonality pertaining to the essence of a singular things in E2p37.

What matters for the notion of lifeform are the mutually maximally useful \dash with respect to both the agents and patients involved \dash activities. The big advantage of an activity-priority notion over a morphology-priority notion of lifeform is that the former has a criterion of identification built in. Morphology-priority notions presumably end up having to rely on extrinsic denominations to produce lifeform groupings. For an activity-priority notion, maximal agreement, in terms of maximal mutual usefulness, serves as the identity criterion. It's non-ad hoc, and it's objective, insofar as it involves what actually is maximally useful.\footnote{In this way, an activity-priority reading allows for robust error conditions in identifying lifeforms: you can be wrong in judging about the identity of a lifeform if you are wrong about what is actually in maximal mutual agreement.}

% On the issue of agreement, E4p35c1 refers back to E4p31c: \textquote[][.]{\emph{Insofar as a thing agrees with our nature, it is necessarily good}}

% \section{Reason and passion}\label{sec:Reason}
% Reason in itself is universal

% Striving insofar as something has adequate ideas and confused ideas (E3p9) – but only adequate ideas count? Clearly not: humans agree in what they provide for each other; $x$'s action to provide $y$ with a cup of coffee is a passion to $y$ – sure, it might be a fully rational thing for $x$, but its passionate aspect for $y$ is as significant a factor in participating in human nature as its rational aspect for $x$. Indeed, without both active and passive aspects, it could not be a participation in human nature (this is still true even if $x$ makes coffee for themselves – see what Spinoza says about something being both an action and a passion for the same subject).

% \section{Complicating the grounding relations}\label{sec:Complicating}

% \section{Priority relations between bodies and activities}\label{sec:ObjPriority}

\section{What about morphology?}\label{sec:ObjMorphology}

There's an intuitive plausibility to grounding lifeforms in morphology that seems to get lost on an activity-priority conception. Informally, we tend to identify such things by their superficial morphology, such that it's possible to be surprised that a whale is mammal rather than a very large fish, or that a pumpkin is a berry while a strawberry is, in fact, not. Of course, \enquote{correct}, scientific, species identification will tend to refer to \enquote{deeper} morphological features\footnote{I say \enquote{tend to} because morphology isn't the only resource used by biologists to make taxonomic distinctions. Etiology is relevant, e.g., and DNA is frequently appealed to (we might want to argue that DNA is just a low-level morphology, but it is not on the phenotypic level that we usually refer to as \enquote{morphology}). Moreover, activities themselves have significant roles to play: \emph{giving birth to live young} vs \emph{laying eggs}, for instance, might supervene on certain morphological features, but it's the activities that get appealed to in making the distinctions, and not the morphology.} that override the misleading superficial morphology \dash thus, a whale is not a fish, because it lacks gills (amongst other reasons). An obvious objection to an activity-priority conception of lifeform is that it fails to preserve our intuitions about the centrality of morphology.

In itself, this is a fairly toothless objection, in that there is little reason to suppose that a conception of lifeform needs to preserve those intuitions; indeed, it's precisely the point here to look at an alternative to the received view. But it does raise the issue of where exactly morphology is supposed to fit into this model.

In a significant sense, it is misleading to say that pumpkins are berries and strawberries are not. To be sure, a pumpkin is a botanical berry, and a strawberry something else (a botanical aggregate fruit). But a pumpkin is very much not a \emph{culinary} berry, and a strawberry very much is. Culinary kinds tend to be functional kinds, classified, at least in part, with reference to how they are used in the context of cooking. Thus, we quite reasonably might think of tomatoes  as vegetables, even when they are commonly known to be a (botanical) fruit. In the terminology I've been using here, we could say that what we call vegetables are those things that agree in \dash that are useful for \dash certain culinary activities, namely, for the most part, playing certain roles in savoury dishes.

If you'll forgive the belaboured analogy, the point here is that there are familiar contexts in which we are perfectly comfortable giving classificatory priority to usefulness and agreement in relation to activities, rather than to morphology. That is, there are contexts in which the activity-priority conception is the intuition-preserving position.

Of course, in the culinary case, the utility that's doing the classificatory work is relevant to the cook and the diner; it's extrinsically denominated with respect to the actual plants in question. The botanical classification is, in principle at least, supposed to work on bases inherent to the plants themselves. To this extent, the culinary case is a disanalogy with the conception of lifeform that's the concern of this chapter. But the source of the disanalogy lies in the conception of the entities involved: in the contexts of current botanical and culinary sciences, we are not treating the entities they deal with as agents, with their own strivings, in their own rights. We are not concerned with, e.g., the utility that tomatoes might provide to other tomatoes.

In which case, it's no wonder that we either stick to our own extrinsic classifications or we look to static features to try to ground the classification: we, in our current context, don't get to appeal to dynamic, quasi-social agreement between tomatoes. That's not the case, however, in Spinoza's system, where everything has its own conatus. As discussed in \Cref{sec:Mutuality}, maximal mutual agreement and usefulness provide a built-in objective ground for the ascription of lifeforms.

A secondary objection might be that an activity-priority conception fails to endorse the scientific correction that a whale is not a fish on the basis of morphological differences. Indeed, whale-ish activities presumably agree far more with the activities of various kinds of fish than they do with those of most mammals: if we were to categorise whales purely on grounds of their activities, it's fairly safe to assume that they would go within the fish group rather than the mammal group.

Interestingly, though, the activity-priority conception does preserve the initial, scientifically naive, intuitions about whales. In fact, despite what I say above, I suspect that it might actually be more accurate to align naive intuitions about whales with activities rather than with morphology: a young child will probably tell you that whales are fish because they live in the sea and swim and breathe underwater, etc.; they probably won't refer to morphological features, such as the presence of fins. Similarly, biologists might well note the oddity of whales as mammals in terms of their activities:
{
  \SetBlockThreshold{2}
  \blockcquote[213]{Simpson:1945}[.]{\textins*{b}ecause of their perfected adaptation to a completely aquatic life, with all its attendant conditions of respiration, circulation, dentition, locomotion, etc., the cetaceans are on the whole the most peculiar and aberrant of mammals}
}
On Simpson's account, whales seem like strange mammals because of what they do, and the ways in which they do it, in accordance with a \textquote{completely aquatic life} \dash we might say that, despite certain morphological similarities with land-dwelling species that we classify as mammalian, whales have an aquatic kind of lifeform.

One upshot that we can see quite clearly from both the culinary and the cetacean cases is that the activities that are salient for a lifeform are multiply realisable through the morphological level. Culinary-vegetable activities can be produced by both botanical-vegetable morphologies and botanical-fruit morphologies; the activities of an aquatic kind of lifeform can be produced by both fish morphologies and mammalian morphologies. That is to say, things can, in principle, maximally agree in their activities while differing significantly in their morphologies.

This means that we can maintain a lifeform without having to be be concerned about exactly where morphologies fall, or where to draw the boundaries, within the continuum of Spinoza's gradualist model. It even allows privation cases to be dealt with elegantly. On a morphology-priority conception, when some putative member $x$ of a lifeform $K$ lacks some definitive morphological feature, retaining $x$ within $K$ is not a straightforward problem to solve.

On an activity-priority conception, however, as long as the activities that $x$ \emph{does} perform maximally agree with those performed by the population of $K$, and as long as the activities performed by the population of $K$ maximally agree with those performed by $x$, then $x$ is still a member of $K$; this holds both if $x$ realises the activities through different means and if $x$ cannot produce certain activities at all \dash the member-of-$K$ relation obtains just in case of maximal agreement with the activities that $x$ does indeed produce. In other words, not every member of a lifeform needs to yield every activity of the lifeform, because lifeform-membership gets maintained through yielding proper subsets of lifeform activities.

\section{Conclusion}\label{sec:Conclusion}
Appeal to lifeforms is an informative and productive means of explaining the activities of things. It's also a means that Spinoza helps himself to, at various points. On the face of it, though, it's not clear that he is entitled to make use of lifeforms, at least not in an explanatory capacity, given the restrictions his system imposes on universals, commonalities in the essences of singular things, and boundaries in the continuum of bodily complexity. What those restrictions amount to, however, is simply a rejection of essentialism in the contexts at stake. If Spinoza does make use of a coherent conception of lifeform, then, it must be a non-essentialist one.

I've argued that Spinoza's system does indeed have the resource for such a conception of lifeform. On this reading, Spinozistic lifeforms needn't, and are not, grounded in some pre-existing essence; rather, they are grounded in the activities that lifeforms perform, and are subject to. This provides a conception of lifeform that does not violate the restrictions of the system, and that has its criterion of identity inbuilt, through the fact of mutual maximal usefulness, which obtains at each member of a lifeform, rather than having to be located in some (impossible) universal entity or some (inadmissible) common feature. This amounts to a conception that fulfils the action-theoretic requirements but circumvents the usual essentialist commitments, being intrinsically mutable\todo{I guess I ought to add more about mutability in the body of the paper} and multiply realisable through other (e.g., morphological) levels.

\printbibliography{}
\end{document}

% Local Variables:
% TeX-engine: xetex
% End:
